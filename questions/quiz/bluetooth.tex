\subsection{Bluetooth protocol and security}

\question{How does Bluetooth Classic handle privacy concerns compared to Bluetooth LE?}
\begin{checkboxes}
    \choice Bluetooth Classic uses encryption keys for all data transmissions.
    \choice Bluetooth Classic limits the number of devices that can connect simultaneously.
    \choice Bluetooth Classic randomizes MAC addresses for improved privacy.
    \CorrectChoice Bluetooth Classic does not have any privacy features.
    \choice None of the other options.
\end{checkboxes}

\question{What is the Bluesnarfing attack?}
\begin{checkboxes}
    \choice A vulnerability that allows unauthorized access to Bluetooth devices via malicious code execution.
    \choice A method for securely pairing Bluetooth devices using encryption keys.
    \choice A feature that enhances Bluetooth data transmission speed.
    \choice A protocol used for secure pairing in Bluetooth devices.
    \CorrectChoice None of the other options.
\end{checkboxes}

\question{Which security services does Bluetooth support?}
\begin{checkboxes}
    \CorrectChoice Authorization
    \CorrectChoice Confidentiality
    \CorrectChoice Authentication
    \choice Notarization
    \choice None of the other options.
\end{checkboxes}


\question{Which multiple access mechanism BT uses?}

\question{What is Bluetooth LE Privacy Feature?}
\begin{checkboxes}
    \choice A feature that encrypts all data transmitted over Bluetooth Low Energy (LE).
    \CorrectChoice A feature that generates random MAC addresses for advertising packets.
    \choice A feature that hides the Bluetooth device from unauthorized scanning.
    \choice A feature that restricts the range of Bluetooth connections to improve privacy.
    \choice None of the other options.
\end{checkboxes}


\question{What is pairing in Bluetooth technology? Select one:}
\begin{checkboxes}
    \choice A method for managing power consumption in Bluetooth devices.
    \CorrectChoice The process of establishing a Bluetooth connection between two devices.
    \choice A mechanism for securely storing Bluetooth device information for future connections.
    \choice None of the other options.
    \choice A protocol for encrypting Bluetooth data transmissions.
\end{checkboxes}


\question{Consider the Bluetooth Authentication sketched in the figure below.}
\begin{solution}
    \begin{enumerate}
        \item A sends its Public Key PKa
        \item B sends its Public Ket PKb
        \item B selects a Random value Nb
        \item A selects a Random value Na
        \item B computes the confirmation Cb = f4(Pkb, Pia, Nb, 0)
        \item B sends the confirmation Cb to A
        \item A sends its nonce Na
        \item B sends its nonce Nb
        \item A checks if Cb = f4(Pkb, Pia, Nb, 0)
    \end{enumerate}
\end{solution}

\question{What is bonding in Bluetooth technology?}
\begin{checkboxes}
    \choice A mechanism for optimizing Bluetooth connection range.
    \choice A method for encrypting data packets during Bluetooth transmission.
    \CorrectChoice A process of permanently storing Bluetooth device information after successful pairing.
    \choice A protocol for managing multiple simultaneous Bluetooth connections.
    \choice None of the other options.
\end{checkboxes}

\question{What is the Relay Attack in Bluetooth?}
\begin{checkboxes}
    \CorrectChoice It bridges communications between devices to fool them into thinking they are close to each other
    \choice It consists in brute forcing the PIN.
    \choice It sends a previously recorded frame to misuse the communication.
    \choice It leverages the signal to noise ratio to authenticate the device.
    \choice None of the other options
\end{checkboxes}

\question{Which are the characteristics the Bluetooth devices must have to support passkey entry?}
\begin{checkboxes}
    \choice Both devices must have a keypad and a display.
    \choice Both devices must have a keypad, a display and a NFC reader.
    \choice It is sufficient that both devices have a display.
    \choice It is sufficient that one device has a keyboard.
    \CorrectChoice None of the other options.
\end{checkboxes}

\question{Associate the correct definition to the correct Bluetooth Association Modes}
\begin{solution}
    % Source: google doc
    \begin{itemize}
        \item \textbf{Passkey Entry}: The user enters the passkey into the device which must have a keyboard.
        \item \textbf{Out-of-Band}: Use a mechanism that is not based on Bluetooth to associate.
        \item \textbf{Numeric Comparison}: A number is displayed on the two devices and the user confirms the number is identical.
        \item \textbf{Just Works}: Used for devices with no Display or Keyboard, with default passkey.
    \end{itemize}
\end{solution}

\question{How is a Man-in-the-Middle (MITM) attack prevented in Bluetooth simple security pairing?}
\begin{checkboxes}
    \choice By rotating the Bluetooth MAC address during the pairing process.
    \choice By using public key cryptography for key exchange.
    \CorrectChoice By encrypting the pairing process using a shared secret key.
    \choice By using a fixed PIN that is known only to the user and the device.
    \choice None of the other options.
\end{checkboxes}

\question{In Bluetooth, the Security Level 3 - Authenticated Pairing with Encryption - guarantees which of the following properties?}
\begin{checkboxes}
    \CorrectChoice Man in The Middle (MITM) protection
    \CorrectChoice Device authenticate required
    \choice Minimal user interaction - the user is not involved during pairing.
    \choice No encryption keys are generated.
    \choice None of the other options.
\end{checkboxes}

\question{Associate the correct definition of the following term as defined in the Bluetooth standards:}
\begin{solution}
    \begin{itemize}
        \item \textbf{Services}: Contains the data related to aspects of the device and supports specified operations on that data
        \item \textbf{Bluetooth Core specification}: Defines the stack layers, protocols and procedures
        \item \textbf{Profiles}: Define the rules for using BT technology for a particular product or application type
        \item \textbf{Characteristics}: Contains the value used in a service along with appropriate permissions
        \item \textbf{Manster and Slave}: Roles devices can take when forming a piconet
    \end{itemize}
\end{solution}

\question{What is the Host Controller Interface (HCl) in Bluetooth technology?}
\begin{checkboxes}
    \choice A standard interface used for communication between the client device and the server device.
    \CorrectChoice A standard interface used for communication between the host device and the Bluetooth module.
    \choice A security protocol for encrypting Bluetooth communication.
    \choice A protocol for data exchange in Bluetooth communication.
    \choice None of the other options.
\end{checkboxes}

\question{Associate the correct definition of the Bluetooth Association Modes}
\begin{solution}
    \begin{itemize}
        \item \textbf{Just works}: Used for devices with no Display or Keyboard, with default passkey.
        \item \textbf{Numeric Comparison}: A number is displayed on the two devices and the user confirms the number is identical.
        \item \textbf{Out-of-Band}: Use a mechanism that is not based on Bluetooth to associate.
        \item \textbf{Passkey Entry}: The user enters the passkey into the device which must have a keyboard.
    \end{itemize}
\end{solution}