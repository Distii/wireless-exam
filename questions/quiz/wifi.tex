\subsection{WiFi protocol and security}

\question{Which of the following topologies are possible in a Wireless LAN?}
\begin{checkboxes}
    \choice Point-to-point mode
    \choice High speed mode
    \CorrectChoice Infrastructure-based mode
    \CorrectChoice Ad-hoc mode
    \choice None of the other options

\end{checkboxes}

\question{Two devices connected in Wi-Fi without RTS/CTS assumed to collide, order what happened (and there where a series of steps like A sends, B, sends, DATA(A) collide with DATA(B) etc,)}

\question{Sort the exchanged messages to connect to an AP}

\question{How does WEP key management compare to WPA/WPA2 key management?}
\begin{checkboxes}
    \choice WEP is more efficient and faster than WPA/WPA2 in managing keys.
    \choice WEP uses dynamic keys that change frequently, unlike WPA/WPA2.
    \CorrectChoice WEP is considered insecure due to its use of static keys, whereas WPA/WPA2 use more secure key management protocols.
    \choice WEP provides stronger encryption and is more secure than WPA/WPA2.
    \choice None of the other options.
\end{checkboxes}

\question{why not csma/ cd in wifi}

\question{authentication service - in the context of wireless networks}

\question{Why is power control important in CDMA networks?}
\begin{checkboxes}
    % source: google doc
    \choice To enhance the data transmission speed and reduce latency.
    \CorrectChoice To minimize interference and improve spectral efficiency.
    \choice To limit interference between adjacent cells.
    \choice To conserve battery life in mobile devices.
    \choice None of the other options.
\end{checkboxes}

\question{Purpose of DIFS in wifi networks}

\question{802.11 network - using ISM BW range, how many independent channels can be used?}

\question{Which wifi protocols are considered secure? (wpa with tkip, wpa2, wep, wpa3 )}

\question{Definition of authenticator supplicant port server authentication}

\question{What is the primary purpose of rate adaptation in WiFi networks?}
\begin{checkboxes}
    \choice To adjust the transmission power of devices based on network conditions.
    \choice To select the optimal data rate for transmitting data over the wireless channel based on the AP transmitter power.
    \choice To select the optimal data rate for transmitting data over the wireless channel based on the distance from the AP.
    \choice To adjust the transmission power of devices dynamically.
    \CorrectChoice None of the others.
\end{checkboxes}

\question{Consider the maximum goodput one could reach with the following technologies. Put the following options in order from the HIGHEST to the LOWEST:}
\begin{solution}
    \begin{enumerate}
        \item 802.11n RTS/CTS disabled, UDP + IP
        \item 802.11n RTS/CTS disabled, TCP + IP
        \item 802.11n RTS/CTS enabled, UDP + IP
        \item 802.11n RTS/CTS enabled, TCP + IP
        \item Fast Ethernet, 100Mb/s
        \item 802.11g RTS/CTS disabled, UDP + IP
        \item 802.11g RTS/CTS disabled, TCP + IP
        \item 802.11g RTS/CTS enabled, UDP + IP
        \item 802.11g RTS/CTS enabled, TCP + IP
    \end{enumerate}
\end{solution}


\question{Associate the correct definition for the following attacks in a WLAN:}
\begin{solution}
    \begin{itemize}
        \item \textbf{WEP Password cracking} → Leverage the reusage of Initialization Vector to create a collision and then break the RC4 key.

        \item \textbf{NAV attack}: → An attack where the attacker manipulates the frame duration value to prevent other devices from accessing the channel.

        \item \textbf{Rouge Access Point}: → A fake AP that impersonates the real AP.

        \item \textbf{De-authentication attack}: → The attacker sends forged de-authentication frames to force a STA to repeat the connection process.

        \item \textbf{Chopchop attack} → The process to recover the plaintext content of a frame by replaying a portion of the original frame.
    \end{itemize}
\end{solution}

\question{Which of the following best describes a NAV attack in WiFi networks?}
\begin{checkboxes}
    \choice An attack where the attacker decrypts the communication between two devices.
    \choice An attack where the attacker sends excessive data packets to the network, causing a denial of service.
    \choice An attack where the attacker sets a very short NAV value to prevent other devices from accessing the channel.
    \CorrectChoice An attack where the attacker manipulates the NAV value to prevent other devices from accessing the channel.
    \choice None of the other options.
\end{checkboxes}


\question{Which of the following information would an attacker require to mount a de-authentication attack against one specific STA in a WLAN?}
\begin{checkboxes}
    \CorrectChoice The AP MAC address
    \CorrectChoice The STA MAC address
    \CorrectChoice The WLAN channel
    \choice The WLAN ESSID
    \choice None of the other options.
\end{checkboxes}


\question{Why in 802.11 there are two destination mac addresses?}
\begin{checkboxes}
    \CorrectChoice The first indicates the AP that has to receive the frame. The second indicates the interface of the router that frame is destined to.
    \choice To correctly identify the STA sending the frame.
    \choice For anonymisation.
    \choice For error correction.
    \choice None of the other options.
\end{checkboxes}

\question{Consider the case a STA would like to transmit a frame to the AP. No other transmissions are present. RTS/CTS are \textit{disabled}. Put in the correct order over time the sequence of events that would occur:}
\begin{solution}
    \begin{enumerate}
        \item A waits for DIFS time
        \item A sends the DATA(A) frame
        \item AP receives the DATA(A) frame
        \item AP waits for SIFS time
        \item AP sends ACK(A)
        \item The transmission is completed
    \end{enumerate}
\end{solution}

\question{Consider the case a STA would like to transmit a frame to the AP. No other transmissions are present. RTS/CTS are \textit{ENABLED}. Put in the correct order over time the sequence of events that would occur:}
\begin{solution}
    \begin{enumerate}
        \item A waits for DIFS time
        \item A sends RTS
        \item AP received RTS and wait SIFS time
        \item AP sends CTS
        \item A receives CTS and waits SIFS time
        \item A sends the DATA(A) frame
        \item AP receives the DATA(A) frame
        \item AP waits for SIFS time
        \item AP sends ACK(A)
        \item the transmission is completed
    \end{enumerate}
    "A receives CTS and waits DIFS time" is not a correct step.
\end{solution}

\question{Why in 802.11 there are two destination mac addresses?}
\begin{checkboxes}
    \choice In case there are multiple APs, to indicate the two APs that should receive and process the frame.
    \choice For error correction
    \choice For anonymisation
    \CorrectChoice None of the other options.
    \choice To correctly identify the STA sending the frame
\end{checkboxes}

\question{What type of encryption does OWE use to secure data in transit?}
\begin{checkboxes}
    \CorrectChoice Diffie-Hellman key exchange to establish a shared secret.
    \choice Asymmetric encryption using RSA keys.
    \choice Symmetric key encryption with a pre-shared key.
    \choice None of the other options.
    \choice Static encryption keys manually configured by the user.
\end{checkboxes}

\question{Consider two STA, A and B, that belong to a WLAN managed by an AP. RTS/CTS are \textit{DISABLED}.
    A and B wake up at the same time to send a data frame, creating thus a collision.
    Put in the right sequence the frames and events that could occur in time.}
\begin{solution}
    \begin{enumerate}
        \item A sends a DATA(A) frame
        \item B sends a DATA(B) frame
        \item DATA(A) collides with DATA(B)
        \item A and B wait for the AP's ACK
        \item Neither A nor B received the ACK, so they backoff
        \item A waits 10ms and tries to retransmit DATA(A)
        \item B waits 20ms and listens to the channel, sensing its busy with DATA(A) transmission
        \item AP receives DATA(A)
        \item AP sends an ACK to A
    \end{enumerate}
\end{solution}

\question{Consider two STA, A and B, that belong to a WLAN managed by an AP. RTS/CTS are \textit{ENABLED}.
    A and B wake up at the same time to send a data frame, creating thus a collision.
    Put in the right sequence the frames and events that could occur in time.}
\begin{solution}
    \begin{enumerate}
        \item A sends a RTS
        \item B sends a RTS
        \item RTS sent by A collides with RTS sent by B
        \item A waits 10ms and tries to retransmit the RTS
        \item B waits 20ms and tries to retransmit the RTS
        \item AP receives the RTS from A and sends a CTS(A)
        \item A and B receive the CTS(AP)
        \item A sends a DATA frame while B defers its transmission
        \item AP sends an ACK to A
    \end{enumerate}
\end{solution}

\question{Why in a WLAN it is not sufficient to use the CSMA-CD protocol used in Ethernet?}
\begin{checkboxes}
    \CorrectChoice It will be impossible to detect an eventual collision by the transmitter because the transmitted signal power would be much stronger than the received signal power.
    \CorrectChoice The WLAN transmitter can either transmit or receive, making it impossible to detect collisions when they occur.
    \choice CSMA-CD requires a bidirectional channel to work. In WLAN there is only one single shared channel.
    \choice CSMA-CD requires an additional channel to signal collisions.
    \choice None of the other options
\end{checkboxes}

\question{Which of the following are effective defenses against deauthentication attacks in WiFi networks?}
\begin{checkboxes}
    \choice Disabling encryption to improve performance.
    \CorrectChoice Honour only de-authentication requests that have the same physical layer properties of other messages sent by such a STA.
    \CorrectChoice Honour de-authentication requests only from STAs that remain idle for the next 15 seconds.
    \CorrectChoice Ignoring de-authentication frames entirely.
    \choice None of the other options.
\end{checkboxes}

\question{Put in the correct order the messages the station and the AP exchange before starting the data exchange:}
\begin{solution}
    \begin{enumerate}
        \item AP send the Beacon
        \item STA sends the Probe request
        \item AP sends the Probe response
        \item STA sends the Authentication request
        \item AP sends the Authentication response
        \item STA sends the Association request
        \item AP sends the Association response
    \end{enumerate}
    Other blocks were not utilized.
\end{solution}

\question{Select the TRUE answers among the following ones about the purposes of CSMA-CA in wireless networking.}
\begin{checkboxes}
    \CorrectChoice It tries to prevent collisions during data transmission.
    \CorrectChoice It ensures all devices have equal access to the wireless medium.
    \CorrectChoice It cannot prevent collisions entirely, especially in high-density networks.
    \choice It encrypts data packets for secure transmission.
    \choice None of the other options.
\end{checkboxes}


\question{Which of the following security features does Simultaneous Authentication of Equals (SAE) provide?}
\begin{checkboxes}
    \choice Stronger message authentication via MIC instead of CRC.
    \CorrectChoice Protection against man-in-the-middle attacks.
    \CorrectChoice Protection against offline dictionary attacks.
    \CorrectChoice Mutual authentication between devices.
    \choice None of the other options.
\end{checkboxes}

\question{What is the primary purpose of the "Duration" field in the WiFi protocol header?}
\begin{checkboxes}
    \choice To indicate the duration of a data transmission.
    \CorrectChoice To reserve the wireless medium for a specified period to avoid collisions.
    \choice To specify the length of time a device can transmit data.
    \choice To define the time interval during which the channel is busy.
    \choice None of the other options.
\end{checkboxes}

\question{How is the value for the "Duration" field typically calculated in WiFi networks?}
\begin{checkboxes}
    \choice Determined by the physical distance between the devices.
    \choice Defined by the number of wireless devices currently connected to the access point.
    \CorrectChoice Calculated according to the duration of the current transmission plus any additional time needed for acknowledgments.
    \choice Based on the data rate of the transmission.
    \choice None of the other options.
\end{checkboxes}

\question{Which of the following provides a possible classification of a wireless communication system?}
\begin{checkboxes}
    \choice High speed networks
    \CorrectChoice Fixed or mobile
    \CorrectChoice Infrastructure or Ad Hoc network
    \CorrectChoice Wireless Wide Area Networks or Wireless Local Area Networks
    \choice None of the other options.
\end{checkboxes}

\question{What is the primary purpose of SIFS (Short Inter-Frame Space) in WiFi networks?}
\begin{checkboxes}
    \choice To synchronize clocks between different wireless devices.
    \choice To prioritize time-sensitive transmissions like acknowledgements (ACK).
    \choice To speed up data communications.
    \choice To encrypt data packets for secure transmission.
    \CorrectChoice None of the other options.
\end{checkboxes}

\question{Which of the following statements correctly describes the difference between deauthentication and disassociation attacks in WiFi networks?}
\begin{checkboxes}
    \choice Deauthentication attacks involve intercepting and decrypting communication, while disassociation attacks involve jamming the wireless signal.
    \CorrectChoice Deauthentication attacks force devices to disconnect by sending fake deauthentication frames, while disassociation attacks force devices to disconnect by sending fake disassociation frames.
    \choice Deauthentication attacks force devices to disconnect by sending fake deauthentication frames, while disassociation attacks force devices to reconnect by sending fake association frames.
    \choice Deauthentication attacks gain unauthorized access by cracking the encryption key, while disassociation attacks flood the network with excessive traffic.
    \choice None of the other options.
\end{checkboxes}


\question{How is the Pairwise Master Key (PMK) typically generated in WPA networks?}
\begin{checkboxes}
    % source: google doc
    \choice It is manually entered by the network administrator.
    \CorrectChoice It is derived from the user's passphrase and other material that makes it unique for each STA connected in the WLAN.
    \choice It is derived from the user's passphrase and other material that makes it identical for all STA connected in the WLAN.
    \choice It is automatically generated by the access point during the initial setup.
    \choice None of the other options.
\end{checkboxes}

\question{What is the hidden terminal problem in WiFi networks?}
\begin{checkboxes}
    % source: google doc
    \choice A situation where a wireless device is physically hidden and cannot connect to the access point.
    \CorrectChoice An issue where wireless devices within the same range of an access point cannot hear each other's transmissions, leading to potential collisions and packet loss.
    \choice A scenario where two wireless devices cannot communicate directly with each other due to obstacles in their line of sight.
    \choice A condition where the access point's signal is too weak for devices to establish a stable connection.
    \choice None of the other options.
\end{checkboxes}

\question{Associate the correct definition of the following terms:}
\begin{solution}
    \begin{itemize}
        \item \textbf{Association request}: A frame the station sends to the AP to enable the communication in the WLAN
        \item \textbf{Authentication request}: A frame the station sends to the AP to provide its identity and request access to the WLAN.
        \item \textbf{Probe request}: A request frame the station sends to look for the presence of a WLAN AP.
        \item \textbf{Beacon}: A broadcast message sent to advertise the presence and capabilities of a WLAN.
    \end{itemize}
\end{solution}

\question{Which of the following statements correctly describes the difference between deauthentication and disassociation attacks in WiFi networks?}
\begin{checkboxes}
    \choice Deauthentication attacks force devices to disconnect by sending fake de-authentication frames, while disassociation attacks force devices to reconnect by sending fake association frames.
    \choice Deauthentication attacks involve intercepting and decrypting communication, while disassociation attacks involve jamming the wireless signal.
    \choice Deauthentication attacks gain unauthorized access by cracking the encryption key, while disassociation attacks flood the network with excessive traffic.
    \CorrectChoice Deauthentication attacks force devices to disconnect by sending fake de-authentication frames, while disassociation attacks force devices to disconnect by sending fake disassociation frames.
    \choice None of the other options.
\end{checkboxes}

\question{Associate the correct definition of the following terms as defined in 802.11x:}
\begin{solution}
    \begin{itemize}
        \item \textbf{Supplicant}: A device that wishes to attach to the LAN/WLAN.
        \item \textbf{Port}: Physical or logical system that allows or blocks the data communication of a device.
        \item \textbf{Authentication Server}: A trusted server that receives and responds to requests for network access, and can tell the authenticator if the connection is to be allowed.
        \item \textbf{Authentication system}: A network device that provides a data link between the client and the network. It can allow or block network traffic between the two.
    \end{itemize}
\end{solution}

\question{What is the primary purpose of Opportunistic Wireless Encryption (OWE)?}
\begin{checkboxes}
    \choice To improve data transmission speed by encrypting data.
    \choice To manage the distribution of encryption keys manually.
    \CorrectChoice To provide data encryption for open networks without user intervention.
    \choice None of the previous options
    \choice To strengthen the security provided by previous protocols by supporting keys up to 256 bits.
\end{checkboxes}