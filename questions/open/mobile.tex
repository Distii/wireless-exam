\subsection{Mobile networks and security}

\question{What are the main security services provided by the SS7 (Signaling System No. 7) standard, and what vulnerabilities does it have? Additionally, how has SS7 evolved to support mobile communications?}

\question{Explain the mutual authentication process in UMTS networks. How does it differ from the authentication process in GSM, and what advantages does it offer?}
\begin{solution}
    Comment: 6/6

    The UTMS ensures mutual autnehtication of the SIM and the network. The main actors of this scheme are. the authentication center contained into the HLR that is the permanent database that allows for the generation and storage of the keys, the VLR that is the temporary database that stores the values send by the HLR in case the SIM changes coverage area and of course the SIM.
    To compute the values uses a shared key K, a random number RAND and together with 5 functions (f1 to f5) will compute all the values. Those values are an authenticated response that will be used to verify the identity of the USIM (XRES), a XMAC and a SQN (sequence number) that are used to guarantee mutual authentication (authentication of the network), three keys: IK, AK and CK that are used for integrity, anonymity and encryption. After generating those number the HLR will forwards them to the VLR (except for the keys that should not be shared) and the VLR will send only the RAND and the AuthN (authentication token, composed by the concatenation between the MAC and the SQN) to the USIM that will use the same functions, key K and RAND to compute the values. First of all will compute the SQN with the f1 function and use this value to compute the MAC and peform those comparison: is the SQN in the right range? is the MAC' generated by the SIM equal to the XMAC computed by the authC? If those two response are positive the authentication of the network is valid. Then, it will compute the other values: keys IK, AK and CK and a SRES'. The value of the SRES' will be send by the SIM to prove its identity to the VLR that will compute if the two SRES and SRES' are equal. If they are equal it means that the SIM is verified. This authentication mechanisms allows to provide mutual autnehtication of both the network and the SIM.
    The GSM doesn't provide mutual authentication and uses a different scheme. The scheme of the GSM only provides the authentication on SIM side provoking several attacks such as stingray attacks that will cause the MS to connect to fake BTS and establishing a connection with them. Also, the keys generated in the GSM are weak and also uses weak algorithms such as A3 (for the signed response) and the A8 (for generating session keys), that was dimonstrated that are weak and an attacker can potential retrieve the ki (authentication key). The key generated for the cipher Kc is only 54 bits long so easy to be attacked. It lacks for integrity. For privacy of the IMSI and avoid to be captured by an attacker it relies on the TMSI. Even if it guarantees privacy it is still easy to compute by an attacker.
    The UTMS allows to have better encryption algorithm, mutual authentication of both the network and the SIM and provides integrity. In respect to the privacy and it still relies on the TMSI even if it easily attackable.
\end{solution}